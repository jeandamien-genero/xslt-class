\documentclass{beamer}
\usepackage[french]{babel}
\usepackage{inputenc}
\usetheme{metropolis}
\usecolortheme{cormorant}
\usepackage{minted}
\usepackage {ifthen}
\newboolean{printCode}
\setboolean{printCode}{true}
% \usemintedstyle{trac}
\usepackage{xcolor}
\usepackage{hyperref}
\usepackage{graphicx}
\usepackage{ulem}  % texte barré

% texte surligné
\usepackage{soul}
\usepackage{color}
\newcommand{\hilight}[1]{\colorbox{lightgray}{#1}}

\title{Technique et chaîne de publication électronique avec XSLT (5/7)}
\date{2024, 3 dec. - 2025, 22 janv.}
\author{Jean-Damien Généro}
\institute{École nationale des chartes -- M2 TNAH}

\begin{document}
    \maketitle

    \section{XSLT. Enregistrement du document de sortie}

    \begin{frame}{XSLT/ Enregistrement : \texttt{<xsl:result-document/>} (1/2)}
        \Large
        \begin{itemize}
            \item \texttt{<xsl:result-document/>} permet de générer plusieurs documents de sortie ; 
            \item Nouveauté de XSLT 2.0 ;
            \item S'utilise comme enfant d'un \texttt{<xsl:template/>} ou d'un \texttt{<xsl:call-template/>} \textit{(cf. cours 6)}.
        \end{itemize}
    \end{frame}

    \begin{frame}{XSLT/ Enregistrement : \texttt{<xsl:result-document/>} (2/2)}
        \Large
        Trois attributs :
        \begin{itemize}
            \item \texttt{@href} permet d'indiquer l'URI du document à générer : absolue (depuis la racine) ou relative (dans ce cas, le processeur part de l'emplacement du fichier d'entrée) ;
            \item \texttt{@method} permet de spécifier le format du code  de sortie ;
            \item \texttt{@indent} permet d'indiquer si le résultat sera indenté ;
        \end{itemize}
    \end{frame}

    \begin{frame}[fragile]{XSLT/ \texttt{<xsl:result-document/>}, exemple}
        \begin{minted}{xml}
<xsl:template match="/">
  <xsl:result-document href="chapitre1.xml"
        method="xml" indent="yes">
    <TEI><!-- règles --></TEI>
  </xsl:result-document>
  <xsl:result-document href="chapitre2.xml"
        method="xml" indent="yes">
    <TEI><!-- règles --></TEI>
  </xsl:result-document>
  <xsl:result-document href="index.xml"
        method="xml" indent="yes">
    <TEI><!-- règles --></TEI>
  </xsl:result-document>
</xsl:template>
        \end{minted}
    \end{frame}

    \begin{frame}{XSLT/ \texttt{<xsl:result-document/>} et variables}
    \Large
    \begin{itemize}
        \item Inconvénient de créer X docs de sortie $\rightarrow$ répéter X fois des éléments récurrents (\texttt{<teiHeader/>}, header HTML, etc.) ;
        \item Stocker ces éléments dans des variables permet de ne les modifier qu'une seule fois.
    \end{itemize}
    \end{frame}

%     \begin{frame}[fragile]{XSLT/ \texttt{<xsl:result-document/>} et URI}
%    \Large
%    \begin{itemize}
%        \item Bonne pratique de programmation $\rightarrow$ déclarer les URI des docs de sortie dans des variables au début de la feuille de style.
%    \end{itemize}
%    \normalsize
%    \begin{minted}{xml}
% <xsl:variable name="chap1">
%        <xsl:value-of select="
%            concat('chapitre1', '.xml')"/>
%</xsl:variable>
%<xsl:template match="/">
%  <xsl:result-document href="$chap1"
%        method="xml" indent="yes">
%    <TEI><!-- règles --></TEI>
%  </xsl:result-document>
%</xsl:template>
%    \end{minted}
%    \end{frame}

    \begin{frame}{XSLT/ \texttt{<xsl:result-document/>}. Notions essentielles}
        \Large
        \begin{itemize}
            \item \texttt{<xsl:result-document/>} permet de générer un ou plusieurs documents de sortie ;
            \bigskip
            \item L'URI est indiquée dans un \texttt{@href} ;
            \bigskip
            \item Les variables sont utiles pour :
            \begin{itemize}
                %\item Déclarer les URI des documents de sortie au début de la feuille de style ;
                \item Contenir les éléments récurrents des documents de sortie (header, footer, etc).
            \end{itemize}
        \end{itemize}
    \end{frame}
    
\end{document}
