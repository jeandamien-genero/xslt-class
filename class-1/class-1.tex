\documentclass{beamer}
\usepackage[french]{babel}
\usepackage{inputenc}
\usetheme{metropolis}
\usecolortheme{cormorant}
\usepackage{minted}
\usepackage {ifthen}
\newboolean{printCode}
\setboolean{printCode}{true}
% \usemintedstyle{trac}
\usepackage{xcolor}
\usepackage{hyperref}
\usepackage{graphicx}
\usepackage{ulem}  % texte barré

% texte surligné
\usepackage{soul}
\usepackage{color}
\newcommand{\hilight}[1]{\colorbox{lightgray}{#1}}

\title{Technique et chaîne de publication électronique avec XSLT}
\date{2025, 1\textsuperscript{er} dec. - 2026, 20 janv.}
\author{Jean-Damien Généro}
\institute{École nationale des chartes -- M2 TNAH}

\begin{document}

  \maketitle
  
  \begin{frame}{Contact}
  \Large
      \begin{itemize}
          \item \textbf{jean-damien.genero@cnrs.fr}
      \end{itemize}
  \end{frame}


    \begin{frame}{Objectifs}
  \Large
      \begin{itemize}
          \item \textbf{XPath}
          \begin{itemize}
              \item Naviguer dans un arbre XML
              \item Manipuler les principales fonctions XPath
          \end{itemize}
          \item \textbf{XSLT}
          \begin{itemize}
              \item Manipuler les règles (\textit{templates}) basiques
              \item Manipuler les conditions et les boucles
          \end{itemize}
          \item \textbf{Édition numérique}
          \begin{itemize}
              \item Transformer un document XML en un autre document XML
              \item Transformer un document XML en un document HTML
              \item Transformer un document XML en un document \LaTeX
              \item Coder une chaîne de publication électronique utilisant XSLT avec Python
          \end{itemize}
      \end{itemize}
  \end{frame}
   

  \section{Écosystème XML}
  \begin{frame}{Écosystème XML/ Principes généraux}
  \Large
  \begin{itemize}
      \item \textbf{XML $\rightarrow$} \textit{langage de balisage qui encode une description de la mise en page et de la structure logique d'un document}
      \begin{itemize}
          \item \textbf{XPath $\rightarrow$} \textit{langage de requête pour naviguer dans la structure hiérarchique d'un document XML à l'aide de chemins} ;
      \end{itemize}
      \bigskip
      \item \textbf{XSL $\rightarrow$} \textit{spécifications pour écrire des feuilles de style} (\textit{stylesheet}) ;
      % \begin{itemize}
          % \item Feuille de style (\textit{stylesheet}) $\rightarrow$ \textit{exprime des intentions sur la manière dont un contenu structuré [i. e. un doc XML] doit être présenté} ;
      % \end{itemize}
      \bigskip
      \item \textbf{XSLT $\rightarrow$} \textit{un langage conçu pour transformer des documents XML en d'autres documents selon les spécifications XSL}.
  \end{itemize}
   
   \footnotesize W3C Recommendations : \href{https://www.w3.org/TR/xml11/}{XML}, \href{https://www.w3.org/TR/xpath-31/}{XPath}, \href{https://www.w3.org/TR/xsl/}{XSL}, \href{https://www.w3.org/TR/xslt-30/}{XSLT}.
   
  \end{frame}
  
  \begin{frame}{Écosystème XML/ Schéma}
      \begin{center}
          \includegraphics[width=8cm, height=7cm]{img/ecosysteme_xml.jpeg}
      \end{center}
  \end{frame}
  
  \begin{frame}{Écosystème XML/ Navigation XPath et de règles XSL}
  \Large
  \begin{itemize}
      \item \textbf{Navigation XPath} :
      \begin{itemize}
          \item Doc XML = structuré, on parle d'\og \textbf{arbre} \fg{} (\textit{XML tree}) ; 
          \item \textbf{Itinéraire} vers une balise (\og \textbf{parent} \fg{} $\rightarrow$ \og \textbf{enfant} \fg{}) ;
          \item Rédaction dans une \textbf{syntaxe propre}. % (langage intégré)
      \end{itemize}
      \bigskip
      \item \textbf{Règle XSL (\textit{template})} :
      \begin{itemize}
          \item \textbf{Agir} sur une ou plusieurs balises (\textit{transform}) ;
          \item Exemples : \textbf{copier} balise + contenu, copier uniquement le contenu et l'\textbf{insérer} dans une nouvelle balise, \textbf{ajouter} ou \textbf{supprimer} un attribut, etc. ;
          \item Rédaction dans une \textbf{syntaxe XML} (langage à balises).
      \end{itemize}
      \bigskip
      \item \textbf{Utilisation avec d'autres langages}
      \begin{itemize}
          \item \texttt{Python}: librairie \texttt{lxml}.
      \end{itemize}
  \end{itemize}
   
  \end{frame}
  
\begin{frame}[fragile]
\frametitle{Écosystème XML/ Exemple d'un chemin XPath dans un \textit{XML tree}}
\begin{itemize}
    \item Comment accéder aux balises \texttt{<p>} avec XPath ?
\end{itemize}
\small
\begin{minted}{xml}
    <TEI>
        <teiHeader/>
        <text>
            <body>
                <div>
                     <p>title</p>
                     <p>text</p>
                </div>
            </body>
        </text>
    </TEI>
\end{minted}
\normalsize
\begin{itemize}
          \item \texttt{<TEI>} $\rightarrow$ \texttt{<text>} $\rightarrow$ \texttt{<body>} $\rightarrow$ \texttt{<div>} $\rightarrow$ \texttt{<p>}
          \item Traduction XPath : \texttt{/TEI/text/body/div/p}
          \item Simplification : \texttt{//body/div/p} (/\sout{TEI/text}/body/div/p)
      \end{itemize}
\end{frame}
  
    \begin{frame}[fragile]
   \frametitle{Écosystème XML/ Exemple d'une règle XSL}
        \begin{itemize}
            \item Comment appliquer une règle XSL aux balises \texttt{<p>} ?
            \begin{itemize}
                \item Une règle XSL est toujours contenue dans une balise \texttt{<xsl:template/>} possédant un \texttt{@match}.
            \end{itemize}
            \item Indiquer le chemin XPath vers les \texttt{<p>} dans  l'\texttt{@match}.
        \end{itemize}
        \bigskip
        \begin{minted}{xslt}
        <xsl:template match="/TEI/text/body/p">
            <xsl:copy-of select="."/>
        </xsl:template>
        \end{minted}
        ou
        \begin{minted}{xslt}
        <xsl:template match="//body/p">
            <xsl:copy-of select="."/>
        </xsl:template>
        \end{minted}
   \end{frame}

    \section{XPath}
    
    \begin{frame}{XPath/ N\oe uds}
        \Large
        \begin{itemize}
            \item \textbf{N\oe ud} (\textit{node}) (7) = composant de l'arbre : \textit{\textbf{root}}, \textbf{\textit{element}}, \textbf{\textit{attribute}}, \textbf{\textit{text}}, \textit{\textbf{comment}}, \textit{\textbf{namespace}} et \textit{\textbf{processing instruction}}.
            \bigskip
            \item Deux rôles : \textbf{\textit{current node}} (fixe, le premier du chemin) ou \textbf{\textit{context node}} (variable, celui que XPath évalue à l'instant \textit{t}) ;
            \bigskip
            \item Écrire un chemin (\textit{expression}) : succession de n\oe uds séparés par l'\textbf{opérateur /} ;
            \begin{itemize}
            \Large
                \item Tester dans Oxygen : \texttt{/TEI/text/body/div/p}
            \end{itemize}
        \end{itemize}
        
    \end{frame}

        \begin{frame}{XPath/ Axes de relation}
        \Large
        \begin{itemize}
            \item Un axe permet de qualifier la \textbf{relation} entre le \textbf{n\oe ud de contexte} et un ou des \textbf{autre(s) n\oe ud(s)}.
            
            \item Direction basique : \textbf{parent $\rightarrow$ enfant}.

            \bigskip

            \item 13 axes XPath, dont :
            \begin{itemize}
            \large
                \item \texttt{parent::} $\rightarrow$ n\oe ud immédiatement au-dessus ;
                \item \texttt{child::} $\rightarrow$ n\oe ud immédiatement en-dessous ;
                \item \texttt{following-sibling::} $\rightarrow$ n\oe ud(s) après celui de contexte et qui ne fait/font pas partie de ses descendants ;
                \item \texttt{ancestor}, \texttt{descendant}, \texttt{preceding}, etc.
            \end{itemize}
        \end{itemize}
        
    \end{frame}

    \begin{frame}{XPath/ Axe \texttt{ancestor::}}
        \begin{center}
            \includegraphics[width=7cm]{img/ancestor.png}
        \end{center}
    \end{frame}

    \begin{frame}{XPath/ Axe \texttt{child::}}
        \begin{center}
            \includegraphics[width=7cm]{img/xslt_child.png}
        \end{center}
    \end{frame}

    \begin{frame}{XPath/ Axe \texttt{following-sibling::}}
        \begin{center}
            \includegraphics[width=9cm]{img/xslt_following_sibling.png}
        \end{center}
    \end{frame}

    \begin{frame}{XPath/ Abbréviations des axes}
    \Large
        \begin{itemize}
            \item 1\up{er} élément de l'arbre (\textit{root node}) $\rightarrow$ \textbf{/} \textit{(attention : différent de la racine \texttt{<TEI>})}
            \item \texttt{decendant-or-self::} $\rightarrow$ \textbf{//}
            \item \texttt{self::} $\rightarrow$ \textbf{.}
            \item \texttt{attribute::} $\rightarrow$ \textbf{@}
            \bigskip
            \item Donc : vous pouvez écrire \texttt{//div/@n} au lieu de \texttt{//div/attribute::n}.
        \end{itemize}
    \end{frame}
    
    \begin{frame}{XPath/ Prédicats : définition}
        \Large
        \begin{itemize}
            \item Prédicat (filtre) = \textbf{condition} qui doit être satisfaite par le n\oe ud de contexte (existence d'un attribut, valeur d'un attribut, position d'une balise, etc.).
            \item Écrit entre crochets \texttt{[ ]}.
            \bigskip
            \item \texttt{//div[@n='2']/head} = le \texttt{<head>} de la \texttt{<div>} avec un \texttt{@n} de valeur \texttt{2}.
            \bigskip
            \item \texttt{//body/div[@n='2']/p[2]} = \textbf{?}
        \end{itemize}
    \end{frame}

    \begin{frame}{XPath/ Prédicats : exemples}
        \Large
        \begin{itemize}
            \item \texttt{//tag[position()=2]} ou \texttt{//tag[2]} $\rightarrow$ condition de position (\texttt{<tag>} n°2) ;
            \bigskip
            \item \texttt{//tag[last()]} $\rightarrow$ dernier \texttt{<tag>} ;
            \bigskip
            \item \texttt{//tag[@foo='bar']} $\rightarrow$ expression logique (\texttt{<tag>} avec un \texttt{@foo} dont la valeur est \texttt{bar}) ;
        \end{itemize}
    \end{frame}

    \begin{frame}{XPath/ Notions essentielles}
        \Large
        \begin{itemize}
            \item Arbre XML, racine, n\oe uds (7, notés avec \texttt{()}) ;
            \item Chemin XPath, opérateur \texttt{/} ;
            \item Axes de relation (13, notés avec \texttt{::}) ;
            \item Prédicats (notés avec \texttt{[]}).
        \end{itemize}
    \end{frame}

    \begin{frame}{XPath/ Exercice. \textit{Journal de Jean Le Fèvre}.}
        \Large
        \begin{itemize}
            \item Donner le chemin de la racine vers \texttt{<msName>} dans le \texttt{<sourceDesc>}.
            \bigskip
            \item Utiliser un prédicat pour sélectionner le \texttt{<idno>} avec le \texttt{@source="gallica"}.
        \end{itemize}
    \end{frame}

    \section{XSLT : élément racine et instructions de premier niveau}

    \begin{frame}{XSLT/ Définition}
        \Large
        \begin{itemize}
            \item XSLT $\rightarrow$ un \textbf{langage de programmation} ;
            \bigskip
            \item Permet de \textbf{transformer un doc XML} en un autre doc  (\texttt{.xml}, \texttt{.html}, \texttt{.tex}, etc.) ;
            \bigskip
            \item Transformation opérée par un \textbf{processeur XSLT}
            \begin{itemize}
                \item \textbf{Construit} l'arbre output, \textbf{transforme} l'arbre input et \textbf{sérialise} le document output ;
                \item Le plus connu $\rightarrow$ \textbf{Saxon} (en ligne de commande) ;
                \item Généralement \textbf{intégré} à des logiciels (\textbf{Oxygen}) ou des librairies (\textbf{python : lxml}).
            \end{itemize}
        \end{itemize}
    \end{frame}

    \begin{frame}{XSLT/ La feuille de style}
    \Large
        \begin{itemize}
            \item Feuille de style XSL == un doc XML \texttt{.xsl} ;
            \item Contient des instructions ou règles (\textit{templates})  ;
            \item Élément racine : \texttt{<xsl:stylesheet>}
        \end{itemize}
        \bigskip
        \bigskip
        \begin{itemize}
            \item \textbf{Dans Oxygen $\rightarrow$ ouvrir un nouveau doc \og XSLT Stylesheet \fg.}
            \item \textbf{Observer les attributs de l'élément racine.}
        \end{itemize}
    \end{frame}

    \begin{frame}{XSLT/ Élément racine : \texttt{<xsl:stylesheet>} (1/4)}
    \Large
    Espaces de nom (\textit{namespace}) et attributs présents de base dans Oxygen :
        \begin{itemize}
            \item \texttt{@xmlns:xsl} : espace de nom XSL ;
            \item \texttt{@xmlns:xs} : espace de nom XML ;
            \item \texttt{@xmlns:math} : espace de nom math (XPath 3) ;
            \item \texttt{@exclude-result-prefixes} : liste des préfixes qui ne seront pas copiés dans l'output ;
            \item \texttt{@version} : version de XSL (1, 2 ou 3).
        \end{itemize}
    \end{frame}

    \begin{frame}{XSLT/ Élément racine : \texttt{<xsl:stylesheet>} (2/4)}
    \Large Attributs à remplacer dans Oxygen \textit{pour une transformation vers XML-TEI}:
        \begin{itemize}
            \item remplacer \texttt{@xmlns:xs} par \texttt{@xmlns:tei} $\rightarrow$ espace de noms de l'élément racine de l'output.
            \begin{itemize}
            \large
                \item évite l'ajout de \texttt{tei:} à chaque élément "matché" par une règle (\texttt{@match}).
            \end{itemize}
            \bigskip
            \item remplacer \texttt{xs} par \texttt{tei} dans \texttt{@exclude-result-prefixes}.
        \end{itemize}
    \end{frame}

    \begin{frame}{XSLT/ Élément racine : \texttt{<xsl:stylesheet>} (3/4)}
    \Large Attributs à ajouter dans Oxygen \textit{pour une transformation vers XML-TEI}:
        \begin{itemize}
            \item \texttt{@xpath-default-namespace} :  espace de noms des chemins XPath de la feuille de style (http://www.tei-c.org/ns/1.0).
            \begin{itemize}
            \large
                \item évite de devoir écrire le préfixe \texttt{tei:} devant chaque n\oe ud d'une expression XPath.
            \end{itemize}
            \bigskip
            \item \texttt{@xmlns} avec l'adresse de l'espace de noms TEI.
            \begin{itemize}
            \large
                \item détermine l'espace de noms de l'ensemble du document de sortie.
            \end{itemize}
        \end{itemize}
    \end{frame}

    \begin{frame}{XSLT/ Élément racine : \texttt{<xsl:stylesheet>} (4/4)}
    \Large
        \begin{itemize}
            \item \textbf{attention} : TEI est un standard XML parmi d'autres ! Pour EAD, changer l'espace de noms TEI par celui d'EAD et le préfixe \texttt{tei:} par \texttt{ead:}.
            \bigskip
            \item Atributs pour la transformation vers HTML ou LaTeX : 
            \begin{itemize}
            \Large
                \item \texttt{@xmlns:xsl} (adresse de l'espace de noms XSLT)
                \item \texttt{@xpath-default-namespace} (adresse de l'espace de noms TEI)
                \item \texttt{@version} (version de XSL utilisée)
            \end{itemize}
        \end{itemize}
    \end{frame}
    
    \begin{frame}[fragile]{XSLT/ Instruction de 1\up{er} niveau : \texttt{<xsl:output>}}
    \Large
        \begin{itemize}
            \item Instruction de premier niveau ;
            \item \og \textbf{contrôle les caractéristiques du document de sortie} \fg.
        \end{itemize}

        \begin{minted}{xml}
<xsl:output
    method="xml | html | text"
    indent="yes | no"
    omit-xml-declaration="yes | no"
    encoding="UTF-8"
/>
        \end{minted}
    \end{frame}

    \begin{frame}[fragile]{XSLT/ Écriture d'une règle : \texttt{<xsl:template>} (1/4)}
        \Large
        \begin{itemize}
            \item une règle peut être écrite de plusieurs manières ;
            \bigskip
            \item \texttt{<xsl:template>} $\rightarrow$ définir une règle ;
            \bigskip
            \item Possède un \texttt{@match} avec pour valeur un chemin XPath qui donne l'emplacement du n\oe ud sur lequel sera appliquée la règle.
            \bigskip
            \item Un doc peut être composé d'une seule règle ou d'une succession de règle. Important $\rightarrow$ \textbf{cohérence des XPath}.
        \end{itemize}
    \end{frame}

    \begin{frame}[fragile]{XSLT/ Écriture d'une règle : \texttt{<xsl:template>} (2/4)}
        \Large
        \begin{itemize}
            \item 1. Appliquer une règle vide sur la racine :
        \end{itemize}
        
        \begin{minted}{xslt}
    <xsl:template match="/">
    </xsl:template>
        \end{minted}
        
        \begin{itemize}
            \item 2. Appliquer la règle \texttt{copy-of} sur la racine :
        \end{itemize}

        \begin{minted}{xslt}
    <xsl:template match="/">
        <xsl:copy-of select="."/>
    </xsl:template>
        \end{minted}
    \end{frame}

    \begin{frame}[fragile]{XSLT/ Écriture d'une règle : \texttt{<xsl:template>} (3/4)}
        \Large
        \begin{itemize}
            \item Effet d'une règle vide $\rightarrow$ la partie de l'arbre d'entrée sélectionnée n'est pas copiée dans l'arbre de sortie.
            \bigskip
            \item Effet de \texttt{<xsl:copy-of/>} $\rightarrow$ copie à l'identique la balise matchée par
le @select et ses n\oe uds enfants ;
            \bigskip
            \item \textbf{Comment copier le \texttt{<text>} dans l'output, mais sans le \texttt{<teiHeader>} ?}
        \end{itemize}
    \end{frame}

    \begin{frame}[fragile]{XSLT/ Écriture d'une règle : \texttt{<xsl:template>} (4/4)}
    \Large
        \begin{itemize}
            \item Comment obtenir cet arbre en output ?
        \end{itemize}
        \normalsize
        \begin{minted}{xml}
    <div>
        <head>Extrait du Journal de
              Jean Le Fèvre</head>
        <div n="1">
            <!-- copie de la div n°1 de l'input -->
        </div>
        <div n="2">
            <!-- copie de la div n°2 de l'input -->
        </div>
    </div>
        \end{minted}
        
    \end{frame}

    \begin{frame}{XSLT : première approche. Notions essentielles}
        \Large
        \begin{itemize}
            \item Transformation et feuille de style XSL ;
            \bigskip
            \item Manipulation de XSL avec Oxygen ;
            \bigskip
            \item Élément racine : \texttt{<xsl:stylesheet>} et ses attributs (en-tête XSL) ;
            \bigskip
            \item Instructions de premier niveau :
            \begin{itemize}
            \Large
                \item \texttt{<xsl:output>} ;
                \item \texttt{<xsl:template>} et son \texttt{@match}.
            \end{itemize}
        \end{itemize}
    \end{frame}
\end{document}
