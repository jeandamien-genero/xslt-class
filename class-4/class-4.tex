\documentclass{beamer}
\usepackage[french]{babel}
\usepackage{inputenc}
\usetheme{metropolis}
\usecolortheme{cormorant}
\usepackage{minted}
\usepackage {ifthen}
\newboolean{printCode}
\setboolean{printCode}{true}
% \usemintedstyle{trac}
\usepackage{xcolor}
\usepackage{hyperref}
\usepackage{graphicx}
\usepackage{ulem}  % texte barré

% texte surligné
\usepackage{soul}
\usepackage{color}
\newcommand{\hilight}[1]{\colorbox{lightgray}{#1}}

\title{Technique et chaîne de publication électronique avec XSLT (4/7)}
\date{2023, 9 janv. - 13 fev.}
\author{Jean-Damien Généro}
\institute{École nationale des chartes -- M2 TNAH}

\begin{document}
    \maketitle

    \section{XSLT. Boucles et tri}

    \begin{frame}{XSLT/ Boucle simple : \texttt{<xsl:for-each/>} (1/2)}
        \Large
        \begin{itemize}
            \item \texttt{<xsl:for-each/>} a un \texttt{@select} ;
            \bigskip
            \item Itération sur chaque n\oe ud désigné par le \texttt{@select}, la règle contenue dans le \texttt{<xsl:for-each/>} est appliquée ;
            \bigskip
            \item Le point de départ des chemins Xpath au sein d'un \texttt{<xsl:for-each/>} est la balise sélectionnée par le \texttt{@select} ;
            \bigskip
            \item Peut intégrer d'autres instructions XSL comme \texttt{<xsl:if/>} et \texttt{<xsl:choose/>}.
        \end{itemize}
    \end{frame}

    \begin{frame}[fragile]{XSLT/ Boucle simple : \texttt{<xsl:for-each/>} (2/2)}
        \Large
        \begin{itemize}
            \item Exemple :
        \end{itemize}
        \begin{minted}{xml}
<xsl:for-each select="//persName">
  <persName>
   <xsl:value-of select="./forename"/>,
   <xsl:value-of select="./surname"/>.
  </persName>
</xs:for-each>
        \end{minted}
    \end{frame}

    \begin{frame}{XSLT/ Tri : \texttt{<xsl:sort/>} (1/2)}
        \Large
        \begin{itemize}
            \item S'utilise comme premier enfant de \texttt{<xsl:apply-templates/>}, \texttt{<xsl:for-each/>} ou \texttt{<xsl:for-each-group/>} ;
            \bigskip
            \item Change l'ordre d'origine des n\oe uds sélectionnés en un autre ordre (alphabétique, numérique, etc.) ;
        \end{itemize}
    \end{frame}

    \begin{frame}[fragile]{XSLT/ Tri : \texttt{<xsl:sort/>} (2/2)}
        \Large
        \begin{itemize}
            \item Exemple :
        \end{itemize}
        \begin{minted}{xml}
<xsl:for-each select="//date[@when]">
  <xsl:sort select="./when" 
            data-type="number"/>
  <xsl:copy-of select="."/>
</xs:for-each>
        \end{minted}
    \end{frame}

    \begin{frame}{XSLT/ Itération sur des groupes : \texttt{<xsl:for-each-group/>} (1/3)}
        \Large
        \begin{itemize}
            % \item "Nouveauté" de XSLT 2.0 ;
            \item Rassemble les n\oe uds (\texttt{@select}) en groupe selon un critère donné (\texttt{@group-by}) et leur applique les règles définies à l'intérieur de \texttt{<xsl:for-each-group/>}.
            \bigskip
            \begin{enumerate}
            \Large
                \item Itère sur les n\oe uds désignés dans le \texttt{@select} ;
                \item Regroupe ces n\oe uds selon le critère défini dans le \texttt{@group-by} ;
                \item Applique les règles indiquées dans le contenu de la balise.
            \end{enumerate}
        \end{itemize}
    \end{frame}

    \begin{frame}{XSLT/ Itération sur des groupes : \texttt{<xsl:for-each-group/>} (2/3)}
        \Large
        \begin{itemize}
            \item Plusieurs \texttt{<xsl:for-each-group/>} sont souvent utilisés ensemble (boucles dans une boucle) ;
            \bigskip
            \item Ils peuvent contenir des \texttt{<xsl:for-each/>} simples, des conditions, etc...
            \bigskip
            \item Attention : c'est une nouveauté de XSLT 2.0.
        \end{itemize}
    \end{frame}

    \begin{frame}{XSLT/ Itération sur des groupes : \texttt{<xsl:for-each-group/>} (3/3)}
        \Large
        \begin{itemize}
            \item \texttt{current-group()} retourne la liste des nœuds du groupe de l'itération en cours d'exécution ;
            \begin{itemize}
            \large
                \item à utiliser dans le \texttt{@select} d'un sous-\texttt{<xsl:for-each-group/>} ou d'un \texttt{<value-of/>}.
            \end{itemize}
            \bigskip
            \item \texttt{current-grouping-key()} retourne la clef utilisée pour l'itération en cours d'exécution.
            \begin{itemize}
            \large
                \item à utiliser dans le \texttt{@select} d'un \texttt{<value-of/>}.
            \end{itemize}
        \end{itemize}
    \end{frame}

    \begin{frame}{XSLT/ Conditions et boucles. Notions essentielles}
        \Large
        \begin{itemize}
            \item Conditions XSL :
            \begin{enumerate}
            \large
                \item Sans alternative : \texttt{<xsl:if/>} (\texttt{@test}) ;
                \item Avec alternative : \texttt{<xsl:choose/>}, contient des \texttt{<xsl:when>} (\texttt{@test}) et un (\texttt{<xs:otherwise/>}).
            \end{enumerate}
            \bigskip
            \item Boucles XSL :
            \begin{enumerate}
            \large
                \item Simple : \texttt{<xsl:for-each/>} (\texttt{@select}) ;
                \item Sur des groupes : \texttt{<xsl:for-each-group/>} (\texttt{@select}, \texttt{@group-by}).
            \end{enumerate}
            \bigskip
            \item Tri XSL : \texttt{<xsl:sort/>} (\texttt{@select}).
        \end{itemize}
    \end{frame}
\end{document}
