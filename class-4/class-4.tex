\documentclass{beamer}
\usepackage[french]{babel}
\usepackage{inputenc}
\usetheme{metropolis}
\usecolortheme{cormorant}
\usepackage{minted}
\usepackage {ifthen}
\newboolean{printCode}
\setboolean{printCode}{true}
% \usemintedstyle{trac}
\usepackage{xcolor}
\usepackage{hyperref}
\usepackage{graphicx}
\usepackage{ulem}  % texte barré

% texte surligné
\usepackage{soul}
\usepackage{color}
\newcommand{\hilight}[1]{\colorbox{lightgray}{#1}}

\title{Technique et chaîne de publication électronique avec XSLT (4/7)}
\date{2023, 9 janv. - 13 fev.}
\author{Jean-Damien Généro}
\institute{École nationale des chartes -- M2 TNAH}

\begin{document}
    \maketitle
    
    \section{XSLT. Conditions}

    \begin{frame}{XSLT/ Conditions}
        \Large
        \begin{itemize}
            \item Deux moyens d'exprimer une condition en XSLT : \texttt{<xsl:if/>} et \texttt{<xsl:choose/>} ;
            \bigskip
            \item Attention : les conditions XSLT ne fonctionnent pas de la même manière que celles d'autres langages de programmation, comme Python où l'on trouve le triplet \texttt{if}, \texttt{elif} et \texttt{else}.
        \end{itemize}
    \end{frame}

    \begin{frame}{XSLT/ \texttt{<xsl:if/>} (1/2)}
        \Large
        \begin{itemize}
            \item \texttt{<xsl:if/>} possède un \texttt{@test} obligatoire ;
            \item Contenu appliqué si le \texttt{@test} est validé.
            \bigskip
            \item Pas de \texttt{elif} : utiliser plusieurs \texttt{<xsl:if/>} ;
            \item Pas de \texttt{else} : utiliser \texttt{<xsl:choose/>}.
        \end{itemize}
    \end{frame}

    \begin{frame}[fragile]{XSLT/ \texttt{<xsl:if/>} (2/2)}
        \Large
        Exemple : dans \textit{St Julien}, ajouter un \texttt{@n} aux \texttt{<p/>} \textbf{si ces derniers sont dans la \texttt{<div/>} \no 2} :
        \normalsize
        \begin{minted}{xml}
<xsl:template match="//div[@n]/p">
    <p>
        <xsl:if test="parent::div/@n = 2">
            <xsl:attribute name="n">
                ...
            </xsl:attribute>
        </xsl:if>
        <xsl:value-of select="."/>
    </p>
</xsl:template>
        \end{minted}
    \end{frame}

    \begin{frame}{XSLT/ \texttt{<xsl:choose/>}}
        \Large
        \begin{itemize}
            \item \texttt{<xsl:choose/>} a pour enfant un ou des \texttt{<xsl:when/>} (req.) et un seul \texttt{<xsl:otherwise/>} (opt.).
            \bigskip
            \item \texttt{<xsl:when/>} a un \texttt{@test} obligatoire (même fonctionnement que \texttt{<xsl:if/>}) ;
            \bigskip
            \item si aucun \texttt{<xsl:when/>} n'est réalisé, alors \texttt{<xsl:otherwise/>} est activé
        \end{itemize}
    \end{frame}

    \section{XSLT. Boucles}

    \begin{frame}{XSLT/ \texttt{<xsl:for-each/>}}
        \Large
        \begin{itemize}
            \item \texttt{<xsl:for-each/>} a un \texttt{@select} ;
            \item pour chaque n\oe uds contenus dans le \texttt{@select}, la règle contenue dans le \texttt{<xsl:for-each/>} est appliquée.
        \end{itemize}
    \end{frame}
\end{document}
