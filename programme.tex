\documentclass{article}
\usepackage[utf8]{inputenc}
\usepackage[T1]{fontenc}
\usepackage[french]{babel}
\usepackage{amsmath}
\usepackage{geometry}
\geometry{a4paper, margin=1in}

\title{\textsc{École nationale des chartes}

\bigskip

Technique et chaîne de publication électronique avec XSLT}
\author{Jean-Damien Généro}
\date{2025, 1\textsuperscript{er} dec. - 2026, 20 janv.}

\begin{document}
\maketitle

\section*{Informations générales}

\begin{itemize}
    \item Cours en ligne : https://github.com/jeandamien-genero/xslt-class
    \item Contact : jean-damien.genero [at] cnrs [.] fr
    \item Évaluations :
    \begin{itemize}
        \item Contrôle continu le 6 janvier 2026.
        \item Devoir final à rendre le 31 mars 2026.
    \end{itemize}
\end{itemize}

\bigskip

\section*{Programme}

\renewcommand{\arraystretch}{1.5}
\begin{center}
    \begin{tabular}[h]{|p{1.5cm} p{9.6cm}}
        \textbf{1\textsuperscript{er} dec.} & \textbf{1.} Introduction. Premières notions de \texttt{XPath} et de \texttt{XSLT}.\\
        \textbf{8 dec.} & \textbf{2.} \texttt{XSLT} : règles basiques.  \\
        \textbf{9 dec.} & \textbf{3.} \texttt{XSLT} : variables, paramètres et boucles. \texttt{XPath} : fonctions. \\
        \textbf{16 dec.} & \textbf{4.} \texttt{XSLT} :  boucles, tri. Génération de plusieurs output. \\
        \textbf{5 janv.} & \textbf{5.} Transformation de \texttt{XML}  vers \texttt{HTML}. \\
        \textbf{6 janv.} & \textbf{!}  Contrôle continu.
        
        \textbf{6.} Transformation de \texttt{XML}  vers \LaTeX \\
        \textbf{20 janv.} & \textbf{7.} Python et \texttt{XSLT}: chaîne de publication électronique. \\
    \end{tabular}
\end{center}

\bigskip

\section*{Objectifs}
      \begin{itemize}
          \item \texttt{XPath}
          \begin{itemize}
              \item Naviguer dans un arbre \texttt{XML}
              \item Manipuler les principales fonctions \texttt{XPath}
          \end{itemize}
          \item \texttt{XSLT}
          \begin{itemize}
              \item Manipuler les règles (\textit{templates}) basiques
              \item Manipuler les variables, conditions et boucles
          \end{itemize}
          \item Édition numérique
          \begin{itemize}
              \item Transformer un document \texttt{XML} en un autre document \texttt{XML}
              \item Transformer un document \texttt{XML} en un document \texttt{HTML}
              \item Transformer un document \texttt{XML} en un document \LaTeX
              \item Utiliser \texttt{XSLT} en synergie avec Python
          \end{itemize}
      \end{itemize}

\end{document}