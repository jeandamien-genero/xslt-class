\documentclass{beamer}
\usepackage[french]{babel}
\usepackage{inputenc}
\usetheme{metropolis}
\usecolortheme{cormorant}
\usepackage{minted}
\usepackage {ifthen}
\newboolean{printCode}
\setboolean{printCode}{true}
% \usemintedstyle{trac}
\usepackage{xcolor}
\usepackage{hyperref}
\usepackage{graphicx}
\usepackage{ulem}  % texte barré

% texte surligné
\usepackage{soul}
\usepackage{color}
\newcommand{\hilight}[1]{\colorbox{lightgray}{#1}}

\title{Technique et chaîne de publication électronique avec XSLT (2/7)}
\date{2023, 9 janv. - 13 fev.}
\author{Jean-Damien Généro}
\institute{École nationale des chartes -- M2 TNAH}

\begin{document}

  \maketitle

  \section{XSLT. Règles basiques}

    \begin{frame}{XSLT/ \texttt{<xsl:template>} (rappels)}
    \Large
        \begin{itemize}
            \item Une règle s'écrit dans un \texttt{<xsl:template>} ;
            \bigskip
            \item L'attribut \texttt{@match} contient le chemin XPath vers la balise concernée par la règle.
        \end{itemize}
    \end{frame}
    
  \begin{frame}[fragile]{XSLT/ Ordre d'application des règles}
    \Large
        \begin{itemize}
            \item L'ordre d'écriture des règles n'a aucune importance ;
            \item XSLT lit chaque n\oe ud du doc XML et applique la règle associée ; s'il n'y a pas de règle, il passe au suivant.
            \item Donc  $\rightarrow$ vous avez la possibilité de vous organisez comme vous le souhaitez dans votre feuille de style.
        \end{itemize}
    \end{frame}

     \begin{frame}{XSLT/ Contenu d'un \texttt{<xsl:template>}}
         \Large
         \begin{itemize}
             \item \texttt{<xsl:template>} peut contenir :
             \begin{itemize}
             \Large
                 \item Balises XML ou HTML (avec leurs @tt) ou commandes \LaTeX ;
                 \item Du texte ;
                 \item Des règles (\textit{templates}) ou variables XSL : \texttt{<xsl:.../>};
             \end{itemize}
         \end{itemize}
     \end{frame}

     \begin{frame}[fragile]{XSLT/ Contenu d'un \texttt{<xsl:template>} : XML, HTML, \LaTeX{} (1/2)}
     \Large
        \begin{itemize}
            \item Des balises peuvent être écrites directement dans un \texttt{<xsl:template>} (idem pour une commande \LaTeX) :
        \end{itemize}
        \normalsize
        % reprendre saint julien en enlevant les balises <head> dans les titres des chapitres ?
         \begin{minted}{xml}
    <xsl:template match="/TEI/text/body">
        <div>
            <head type="chap">
                <xsl:value-of select="./p[1]"/>
            </head>
            <xsl:copy-of select="./p[2]"/>
        </div>
    </xsl:template>
         \end{minted}
     \end{frame}

     \begin{frame}[fragile]{XSLT/ Contenu d'un \texttt{<xsl:template>} : XML, HTML, \LaTeX{} (2/2)}
     \Large
        \begin{itemize}
            \item Les balises XML ou HTML peuvent être écrites dans des \texttt{<xsl:element name="<tag>">} et les attributs dans des \texttt{<xsl:attribute name="<@tt>">} :
        \end{itemize}
        \normalsize
        % reprendre saint julien en enlevant les balises <head> dans les titres des chapitres ?
         \begin{minted}{xml}
<xsl:template match="/TEI/text/body">
    <xsl:element name="div">
        <xsl:element name="head">
            <xsl:attribute name="type">chap</xsl:attribute>
            <xsl:value-of select="./p[1]"/>
        </xsl:element>
        <xsl:copy-of select="./p[2]"/>
    </xsl:element>
</xsl:template>
         \end{minted}
     \end{frame}

     \begin{frame}[fragile]{XSLT/ Contenu d'un \texttt{<xsl:template>} : du texte (1/2)}
     \Large
        \begin{itemize}
            \item Du texte peut être mis dans une règle : pour remplacer le contenu d'une balise, pour écrire le préambule d'un doc \LaTeX :
        \end{itemize}
        \normalsize
        % reprendre saint julien en enlevant les balises <head> dans les titres des chapitres ?
         \begin{minted}{xml}
        <xsl:template match="/">
            \documentclass[a4paper]{book}
            \usepackage[utf8]{inputenc}
            \usepackage[french]{babel}
            \usepackage{fontspec}
            \begin{document}
                <xsl:apply-templates/>
            \end{document}
        </xsl:template>
         \end{minted}
     \end{frame}

      \begin{frame}[fragile]{XSLT/ Contenu d'un \texttt{<xsl:template>} : du texte (2/2)}
     \Large
     % attention avec <xsl:text> les sauts de ligne sont pris en compte
        \begin{itemize}
            \item Du texte peut aussi être mis dans un \texttt{<xsl:text>text</xsl:text>} :
        \end{itemize}
        \normalsize
         \begin{minted}{xml}
<xsl:template match="/TEI/text/body">
    <xsl:element name="body">
        <xsl:element name="head">
            <xsl:attribute name="type">chap
                </xsl:attribute>
            <xsl:value-of select="./p[1]"/>
        </xsl:element>
        <xsl:element name="p">
            <xsl:text>nouveau texte</xsl:text>
        </xsl:element>
    </xsl:element>
</xsl:template>
         \end{minted}
     \end{frame}

     \begin{frame}{Règles basiques : notions essentielles}
         \Large
         \begin{itemize}
             \item Les balises XML et HTML, et les commandes \LaTeX{} peuvent être écrites en clair dans un \texttt{<xsl:template>} ;
             \item Les balises XML et HTML peuvent aussi être écrites dans un \texttt{<xsl:element name="tag">} et un attribut dans un \texttt{<xsl:attribute name="@att">}.
             \item Le texte peut être mis soit en clair soit dans un \texttt{<xsl:text>}.
         \end{itemize}
     \end{frame}

     \section{Les quatre principales règles XSL}

     \begin{frame}{XSLT/ Les quatre principales règles XSL}
         \Large
         \begin{itemize}
             \item \texttt{<xsl:apply-template/>}
             \bigskip
             \item \texttt{<xsl:copy/>}
             \bigskip
             \item \texttt{<xsl:copy-of/>}
             \bigskip
             \item \texttt{<xsl:value-of/>}
         \end{itemize}
     \end{frame}

     \begin{frame}{XSLT/ Principales règles (1/4) : \texttt{<xsl:apply-template/>}}
         \Large
         \begin{itemize}
             \item \texttt{<xsl:apply-template/>} est une \textbf{instruction récursive} grâce à laquelle les \textbf{règles} définies dans la feuille de style \textbf{sont appliquées aux enfants du n\oe ud courant}.
            \bigskip
            \item Si elle n'est pas présente, le processeur s'arrête à l'emplacement désigné par le \texttt{@match} du \texttt{<xsl:template/>} et ne traite pas les éléments enfants.
            \item Exemple...
         \end{itemize}
     \end{frame}

     \begin{frame}[fragile]{XSLT/ Principales règles (2/4) : \texttt{<xsl:copy>}}
     \Large
     \begin{itemize}
         \item \textbf{Copie de la balise du n\oe ud courant, sans les \textit{namespaces}, attributs, texte, etc.}
         \item \texttt{<xsl:copy/>} peut contenir d'autres règles, du texte, etc.
         \item Intérêt $\rightarrow$ copier un élément pour appliquer des règles à ses enfants.
     \end{itemize}
     \begin{minted}{xml}
<xsl:template match="/TEI/text/body">
    <xsl:copy>...</xsl:copy>
</xsl:template>
     \end{minted}
     \end{frame}

% expliquer la différence entre le noeud courant et le noeud de contexte ?

     \begin{frame}[fragile]{XSLT/ Principales règles (3/4) : \texttt{<xsl:copy-of>}}
     \Large
     \begin{itemize}
         \item \textbf{Copie à l'identique de l'ensemble du n\oe ud courant et de ses n\oe uds enfants (balises, attributs, textes).}
         \item Le n\oe ud courant est toujours contenu dans le \texttt{@select}.
         \item Intérêt $\rightarrow$ reproduire rapidement une partie de l'arbre que l'on ne veut pas modifier.
     \end{itemize}
     \begin{minted}{xml}
<xsl:template match="/TEI/text/body">
    <xsl:copy-of select="."/>
</xsl:template>
     \end{minted}
     \end{frame}

     \begin{frame}[fragile]{XSLT/ Principales règles (4/4) : \texttt{<xsl:value-of>}}
     \Large
     \begin{itemize}
         \item \textbf{Renvoie uniquement la valeur textuelle du n\oe ud courant.}
         \item Le n\oe ud courant est toujours contenu dans le \texttt{@select}.
         \item Intérêt $\rightarrow$ copier du texte sans les balises (utile pour \LaTeX).
     \end{itemize}
     \begin{minted}{xml}
<xsl:template match="/TEI/text/body">
    <xsl:value-of select="."/>
</xsl:template>
     \end{minted}
     \end{frame}

     \begin{frame}[fragile]{XSLT/ Exercice. Combinaison des règles}
     \Large
     \begin{itemize}
         \item Quelle différence entre ces templates ?
     \end{itemize}
     \begin{minted}{xml}
<!-- 1 -->
<xsl:template match="//body/p">
    <xsl:copy-of select="."/>
</xsl:template>
<!-- 2 -->
<xsl:template match="//body/p">
    <xsl:copy>
        <xsl:value-of select="."/>
    </xsl:copy>
</xsl:template>
     \end{minted}
     \end{frame}

     \begin{frame}{XSLT/ Une autre règle : \texttt{<xsl:number/> (1/2)}}
         \Large
         \begin{itemize}
             \item \texttt{<xsl:number/>} \og compte des éléments de façon séquentielle \fg.
             \bigskip
             \item Attributs :
             \begin{itemize}
             \Large
                 \item \texttt{@count} $\rightarrow$ définit les éléments de l'input qui seront numérotés dans l'output ;  % c'est une expression XPath
                 \item \texttt{@level="single|multiple|any"} $\rightarrow$ niveau de l'arbre pris en compte pour le comptage ;
                 \item \texttt{@format="1|01|A|a|I|i"} $\rightarrow$ format des numéros.
             \end{itemize}
         \end{itemize}
     \end{frame}

     \begin{frame}[fragile]{XSLT/ Une autre règle : \texttt{<xsl:number/> (2/2)}}
         \Large
         \begin{minted}{xml}
<xsl:template match="//body/p">
    <xsl:copy>
        <xsl:attribute name="n">
            <xsl:number
                count="//body/p" 
                level="multiple" 
                format="a"/>
        </xsl:attribute>
        <xsl:value-of select="."/>
    </xsl:copy>
</xsl:template>
         \end{minted}
     \end{frame}

     \begin{frame}{XSLT/ Exercice}
         \Large
         $\rightarrow$ Écrire une feuille de style XSL avec le fichier de St Julien l'Hospitalier :
         \begin{itemize}
             \item Constituer un fichier XML-TEI valide ;
             \item Numéroter dans des formats différents les \texttt{<head>} et les \texttt{<p>} ;
             \item Ajouter un nouveau paragraphe ;
             \item Transformer les \texttt{<hi rend="b"/>} en des \texttt{<persName>} et les \texttt{<hi rend="i"/>} en des \texttt{<placename>}.
         \end{itemize}
     \end{frame}
     
\end{document}
