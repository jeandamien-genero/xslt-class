\documentclass{beamer}
\usepackage[french]{babel}
\usepackage{inputenc}
\usetheme{metropolis}
\usecolortheme{cormorant}
\usepackage{minted}
\usepackage {ifthen}
\newboolean{printCode}
\setboolean{printCode}{true}
% \usemintedstyle{trac}
\usepackage{xcolor}
\usepackage{hyperref}
\usepackage{graphicx}
\usepackage{ulem}  % texte barré

% texte surligné
\usepackage{soul}
\usepackage{color}
\newcommand{\hilight}[1]{\colorbox{lightgray}{#1}}

\title{Technique et chaîne de publication électronique avec XSLT (2/7)}
\date{2023, 5 dec. - 2024, 23 janv.}
\author{Jean-Damien Généro}
\institute{École nationale des chartes -- M2 TNAH}

\begin{document}

  \maketitle

  \section{XSLT. Règles basiques}

    \begin{frame}{XSLT/ \texttt{<xsl:template>}}
    \Large
        \begin{itemize}
            \item Un \texttt{<xsl:template>} contient une ou plusieurs règles ;
            \bigskip
            \item L'\texttt{@match} contient le chemin XPath vers la balise qui sera le point de départ de la ou des règles, qui peut s'appliquer à :
            \bigskip
            \begin{itemize}
            \large
                \item La balise désignée dans le \texttt{@match} ;
                \item Les enfants ou descendants de cette balise ;
                \item N'importe quelle balise de l'arbre \textit{via} des axes (\textit{ancestor}, \textit{preceding}, etc.).
                \bigskip
                \item La partie de l'arbre sur laquelle la règle s'applique est en général indiquée dans le \texttt{@select} de l'instruction XSL (\textit{cf. infra}). % même si c'est la même que celle contenue dans le @match
            \end{itemize}
        \end{itemize}
    \end{frame}

     \begin{frame}{XSLT/ Contenu d'un \texttt{<xsl:template>}}
         \Large
         \begin{itemize}
             \item \texttt{<xsl:template>} peut contenir :
             \begin{itemize}
             \Large
                 \item Des balises XML ou HTML (avec leurs @tt) ;
                 \item Du texte (commandes \LaTeX) ;
                 \item Des instructions ou variables/param XSL : \texttt{<xsl:.../>};
             \end{itemize}
         \end{itemize}
     \end{frame}

     \begin{frame}{Contenu d'un \texttt{<xsl:template>} : fichier d'exemple}
         \Large
         \begin{itemize}
            \item Ouvrir Oxygen.
             \item Ouvrir \texttt{/class-2/class2\_teiAll\_oxygen.xml}.
             \item Ouvrir une feuille de style XSLT vide.
             \item Compléter l'en-tête XSLT et ajouter l'instruction \texttt{<xsl:output/>} (\textit{cf.} fiche xsl\_header).
         \end{itemize}
     \end{frame}

     \begin{frame}[fragile]{XSLT/ Contenu d'un \texttt{<xsl:template>} : XML, HTML, \LaTeX{} (1/2)}
     \Large
        \begin{itemize}
            \item Des balises peuvent être écrites directement dans un \texttt{<xsl:template>} :
        \end{itemize}
        \normalsize
        % reprendre saint julien en enlevant les balises <head> dans les titres des chapitres ?
         \begin{minted}{xml}
<xsl:template match="//body">
    <body>
        <div>
            <head type="chap">
                <xsl:value-of select="./p[1]"/>
            </head>
            <xsl:copy-of select="./p[2]"/>
        </div>
    </body>
</xsl:template>
         \end{minted}
     \end{frame}

     \begin{frame}[fragile]{XSLT/ Contenu d'un \texttt{<xsl:template>} : XML, HTML, \LaTeX{} (2/2)}
     \Large
        \begin{itemize}
            \item Les balises XML ou HTML peuvent être écrites dans des \texttt{<xsl:element name="tag">} et les attributs dans des \texttt{<xsl:attribute name="@tt">} :
        \end{itemize}
        \normalsize
        % reprendre saint julien en enlevant les balises <head> dans les titres des chapitres ?
         \begin{minted}{xml}
<xsl:template match="//body">
    <xsl:element name="div">
        <xsl:element name="head">
            <xsl:attribute name="type">chap</xsl:attribute>
            <xsl:value-of select="./p[1]"/>
        </xsl:element>
        <xsl:copy-of select="./p[2]"/>
    </xsl:element>
</xsl:template>
         \end{minted}
     \end{frame}

     \begin{frame}[fragile]{XSLT/ Contenu d'un \texttt{<xsl:template>} : du texte (1/2)}
     \Large
        \begin{itemize}
            \item Du texte peut être mis dans une règle : pour remplacer le contenu d'une balise, pour écrire le préambule d'un doc \LaTeX.
            \item (Dans le fichier d'exemple : remplacer tout.)
        \end{itemize}
        \normalsize
        % reprendre saint julien en enlevant les balises <head> dans les titres des chapitres ?
         \begin{minted}{xml}
        <xsl:template match="/">
            \documentclass[a4paper]{book}
            \usepackage[utf8]{inputenc}
            \usepackage[french]{babel}
            \usepackage{fontspec}
            \begin{document}
                <xsl:apply-templates/>
            \end{document}
        </xsl:template>
         \end{minted}
     \end{frame}

      \begin{frame}[fragile]{XSLT/ Contenu d'un \texttt{<xsl:template>} : du texte (2/2)}
     \Large
        \begin{itemize}
            \item Du texte peut aussi être mis dans un \texttt{<xsl:text>text</xsl:text>} :
        \end{itemize}
        \normalsize
         \begin{minted}{xml}
<xsl:template match="/TEI/text/body">
    <xsl:element name="body">
        <xsl:element name="head">
            <xsl:attribute name="type">chap</xsl:attribute>
            <xsl:value-of select="./p[1]"/>
        </xsl:element>
        <xsl:element name="p">
            <xsl:text>nouveau texte</xsl:text>
        </xsl:element>
    </xsl:element>
</xsl:template>
         \end{minted}
     \end{frame}

     \begin{frame}{Règles basiques : notions essentielles}
         \Large
         \begin{itemize}
             \item Les balises XML et HTML peuvent être écrites en clair dans un \texttt{<xsl:template>} ;
             \item Les balises XML et HTML peuvent aussi être écrites dans un \texttt{<xsl:element name="tag">} et un attribut dans un \texttt{<xsl:attribute name="@att">}.
             \item Le texte (commandes \LaTeX{}) peut être mis soit en clair soit dans un \texttt{<xsl:text>}.
         \end{itemize}
     \end{frame}

     \section{Les quatre principales instructions XSL}

     \begin{frame}{XSLT/ Les quatre principales instructions XSL}
         \Large
         \begin{itemize}
             \item \texttt{<xsl:copy/>}
             \bigskip
             \item \texttt{<xsl:copy-of/>}
             \bigskip
             \item \texttt{<xsl:value-of/>}
             \bigskip
             \item \texttt{<xsl:apply-templates/>}
         \end{itemize}
     \end{frame}
     
     \begin{frame}[fragile]{XSLT/ Principales instructions (1/4) : \texttt{<xsl:copy>}}
     \Large
     \begin{itemize}
         \item \textbf{Copie de la balise matchée par le \texttt{@match}, sans les \textit{namespaces}, attributs, texte, etc.}
         \item \texttt{<xsl:copy/>} peut contenir d'autres règles, du texte, etc.
         \item Intérêt $\rightarrow$ copier un élément pour appliquer des règles à ses enfants.
     \end{itemize}
     \begin{minted}{xml}
<xsl:template match="/TEI/text/body">
    <xsl:copy>...</xsl:copy>
</xsl:template>
     \end{minted}
     \end{frame}

     \begin{frame}[fragile]{XSLT/ Principales instructions (2/4) : \texttt{<xsl:copy-of>}}
     \Large
     \begin{itemize}
         \item \textbf{Copie à l'identique de la balise matchée par le \texttt{@select} et de ses n\oe uds enfants (balises, attributs, textes).}
         \item Élément vide (pas de règles internes).
         \item \textbf{Impossible de modifier les éléments copiés !}
         \item Intérêt $\rightarrow$ reproduire rapidement une partie de l'arbre que l'on ne veut pas modifier.
     \end{itemize}
     \begin{minted}{xml}
<xsl:template match="/TEI/text/body">
    <xsl:copy-of select="."/>
</xsl:template>
     \end{minted}
     \end{frame}

     \begin{frame}[fragile]{XSLT/ Principales instructions (3/4) : \texttt{<xsl:value-of>}}
     \Large
     \begin{itemize}
         \item \textbf{Renvoie uniquement la valeur textuelle de la balise matchée par le \texttt{@select}}.
         \item Élément vide (pas de règles internes).
         \item Intérêt $\rightarrow$ copier du texte sans les balises (utile pour \LaTeX).
     \end{itemize}
     \begin{minted}{xml}
<xsl:template match="/TEI/text/body">
    <xsl:value-of select="."/>
</xsl:template>
     \end{minted}
     \end{frame}

    \begin{frame}[fragile]{XSLT/ Exercice. Combinaison des instructions}
     \Large
     \begin{itemize}
         \item Quelle différence entre ces templates ?
     \end{itemize}
     \begin{minted}{xml}
<!-- 1 -->
<xsl:template match="//body/p">
    <xsl:copy-of select="."/>
</xsl:template>
<!-- 2 -->
<xsl:template match="//body/p">
    <xsl:copy>
        <xsl:value-of select="."/>
    </xsl:copy>
</xsl:template>
     \end{minted}
     \end{frame}

    \begin{frame}{XSLT/ Principales instructions (4/4) : \texttt{<xsl:apply-templates/>}}
         \Large
         \begin{itemize}
             \item \texttt{<xsl:apply-templates/>} est une \textbf{instruction récursive} : le processeur va \textbf{examiner les n\oe uds enfants} de la balise  matchée par le \texttt{@match} dans l’ordre et \textbf{appliquer les règles} qui leur sont associées.
            % pour faire deux règles différentes :
            % (1) reproduire la balise <TEI>
            % (2) appliquer les règles sur body des slides précédentes :
            % <xsl:template match="/">
            %   <TEI>
            %       <xsl:apply-templates/> --> permet de dire "applique des règles aux éléments enfants"
            %   </TEI>
            % </xsl:template>
         \end{itemize}
     \end{frame}


    \begin{frame}[fragile]{XSLT/ L'élément \texttt{<xsl:apply-templates/>} (1/4)}
        \Large
        \begin{itemize}
            \item Sans elle, le processeur s'arrête à l'emplacement désigné par le \texttt{@match} du \texttt{<xsl:template/>} et ne traite pas les éléments enfants.
            \item Exemple : la balise \texttt{<TEI/>}.
        \end{itemize}
        % noeud courant = celui dans @select (fixe)
        % noeud de contexte = celui sur le quel opère la règle (variable)
        \begin{minted}{xml}
<xsl:template match="/">
    <TEI>
        <xsl:apply-templates/>
    </TEI>
</xsl:template>
        \end{minted}
    \end{frame}

    \begin{frame}[fragile]{XSLT/ L'élément \texttt{<xsl:apply-templates/>} : \texttt{@select} (2/4)}
        \Large
        \begin{itemize}
            \item \texttt{@select} permet d'appliquer les règles uniquement au n\oe ud sélectionné.
            \bigskip
            \item Ex. $\rightarrow$ inverser \texttt{<teiHeader/>} et \texttt{<text/>} (\textit{cf.} fichier) : 
        \end{itemize}
        \normalsize
        \begin{minted}{xml}
<xsl:template match="/">
    <TEI>
        <xsl:apply-templates select="//text"/>
        <xsl:apply-templates select="//teiHeader"/>
    </TEI>
</xsl:template>
        \end{minted}
    \end{frame}

    \begin{frame}{XSLT/ L'élément \texttt{<xsl:apply-templates/>} : \texttt{@mode} (3/4)}
        \Large
        \begin{itemize}
            \item \texttt{@mode} permet d'appliquer des règles différentes \textbf{à un même élément XML} en fonction de son emplacement ou de son contenu (= son \og mode \fg).
            \bigskip
            \item Le \texttt{@mode} doit être présent et dans le \texttt{<xsl:apply-templates/>} et dans le \texttt{<xsl:template/>} avec la même valeur (= le nom du mode).
        \end{itemize}
    \end{frame}

    \begin{frame}[fragile]{XSLT/ L'élément \texttt{<xsl:apply-templates/>} : \texttt{@mode} (4/4)}
        \Large
        \begin{itemize}
            \item Exemple d'utilisation de \texttt{@mode} : créer une table des matières (\textit{cf.} fichier).
        \end{itemize}
        \normalsize
        \begin{minted}{xml}
<xsl:template match="//body">
    <xsl:copy>
        <xsl:apply-templates />
        <div>
            <head>Table des matières</head>
            <xsl:apply-templates mode="toc"/>
        </div>
    </xsl:copy>
</xsl:template>
        \end{minted}
    \end{frame}
     
     \begin{frame}{XSLT/ Une autre instruction : \texttt{<xsl:number/> (1/2)}}
         \Large
         \begin{itemize}
             \item \texttt{<xsl:number/>} \og compte des éléments de façon séquentielle \fg.
             \bigskip
             \item Attributs :
             \begin{itemize}
             \Large
                 \item \texttt{@count} $\rightarrow$ définit les éléments de l'input qui seront numérotés dans l'output ;  % c'est une expression XPath
                 \item \texttt{@level="single|multiple|any"} $\rightarrow$ niveaux de l'arbre pris en compte pour le comptage ;
                 \item \texttt{@format="1|01|A|a|I|i"} $\rightarrow$ format des numéros.
             \end{itemize}
         \end{itemize}
     \end{frame}

     \begin{frame}[fragile]{XSLT/ Une autre instruction : \texttt{<xsl:number/> (2/2)}}
         \Large
         \begin{minted}{xml}
<xsl:template match="//body//p">
        <xsl:copy>
            <xsl:attribute name="n">
                <xsl:number
                    count="//div[@n]/p" 
                    level="any" 
                    format="1"/>
            </xsl:attribute>
            <xsl:value-of select="."/>
        </xsl:copy>
</xsl:template>
         \end{minted}
     \end{frame}

     \begin{frame}{XSLT/ Ordre d'application des règles}
    \Large
        \begin{itemize}
            \item XSLT commence par chercher la règle à appliquer au n\oe ud racine ;
            \item Cette règle fait appel à d'autres avec \texttt{apply-templates} (avec ou sans @tt), elles sont appliquées dans l'ordre ;
            \item S'il n'y a pas de règle, il passe au suivant.
            \bigskip
            \item Donc : l'ordre d'écriture des règles n'a aucune importance ;
            \item \textbf{Attention  $\rightarrow$ restez lisible + commentez votre code !}
        \end{itemize}
    \end{frame}

     \begin{frame}{XSLT/ Exercice}
        \Large
         $\rightarrow$ Écrire une feuille de style XSL avec le fichier de \textit{St Julien l'Hospitalier} :
         \normalsize
         \begin{itemize}
         \large
             \item Constituer un fichier XML-TEI valide ;
             \item Ajouter un \texttt{<editionStmt/>} (avec \texttt{<edition/>} et \texttt{<respStmt/>}) ;
             \item Numéroter dans des formats différents les \texttt{<head>} et les \texttt{<p>} ;
             \item Transformer les \texttt{<hi rend="b"/>} en des \texttt{<persName>} et les \texttt{<hi rend="i"/>} en des \texttt{<placename>} ;
             \item Bonus : ajouter le début du chap. 3 (lien dans le \texttt{<teiHeader/>}).
         \end{itemize}
     \end{frame}
     
\end{document}
