\documentclass[]{book}
        \usepackage[utf8]{inputenc}
        \usepackage[T1]{fontenc}
        \usepackage[a4paper]{geometry}
        \usepackage[french]{babel}
        
        \usepackage{titlesec}
        \titleformat{\chapter}[block]{\Large\bfseries}{}{1em}{}
        
        \usepackage[splitindex]{imakeidx}
        \makeindex[name=indiv,title=Index des noms de personnes]
        \makeindex[name=lieux,title=Index des noms de lieux]
        
        \usepackage[hidelinks, pdfstartview=FitH, plainpages=false]{hyperref}
        
        \title{Journal de Jean Le Fèvre, évêque de Chartres, chancelier des rois de Sicile Louis I et Louis II d'Anjou}
        \author{Jean Le Fèvre}
        \date{1381-1388}
        
        \begin{document}
        \maketitle
        
        %% Votre code ici
        
        \chapter*{Informations sur le manuscrit}
        \addcontentsline{toc}{chapter}{Informations sur le manuscrit}
        
        \section*{Institution de conservation}
        \addcontentsline{toc}{section}{Institution de conservation}
        
        \begin{itemize}
            \item Institution: Bibliothèque nationale de France
            \item Cote: ms fr. 5015
            \item Numérisation: \href{https://gallica.bnf.fr/ark:/12148/btv1b9007462z}{Gallica}
        \end{itemize}
        
        \section*{Description du manuscrit}
        \addcontentsline{toc}{section}{Description du manuscrit}
        
        \begin{itemize}
            \item Support: papier
            \item Feuillet: 218 feuillets
            \item Note: Le premier feuillet a été, par suite d'une erreur de reliure, placé à la fin du volume. Le texte commence, sur ce feuillet, mutilé dans le haut, par : "L'an mil CCC quatre vins et un, le mecredi derrenier jour de juillet...", et se continue au fol. 2, placé en tête du volume, par : "De par le duc d'Anjou et de Touraine. Chancellier, savoir vous faisons...". Il finit (fol. 218 v°) par : "... cui domino concedit officium jumelle, alias dictum la publicane civitatis neapolitane". Les premiers mots que nous avons donnés pour le fol. 2, sont ceux d'une pièce transcrite dans le journal. Le texte même du journal commence au bas du feuillet par : "Lundi, XXII jour de septembre, l'an IIIIxx et un...". La majeure partie de ce manuscrit paraît autographe.
        \end{itemize}
        
        
            \chapter*{1. Jean Le Fèvre apprend la mort de Louis Ier}
            
            \addcontentsline{toc}{chapter}{1. Jean Le Fèvre apprend la mort de Louis Ier}
            
            Le XXVI jour dudit mois assés pres d'Angiers\index[lieux]{Angiers} en venant, je encontré Guillaume de Nades\index[indiv]{Nades (Guillaume de)} qui me dit la mort de monsegneur le roy Loys\index[indiv]{Anjou (Louis Ier)}, laquelle fu le XX jour du mois precedent, a Bar\index[lieux]{Bar}; et l'avoit monseigneur de Berri\index[indiv]{Berry (Jean de)} envoié au conseil de Madame\index[indiv]{Blois (Marie de)} pour leur dire qu'il ne le feissent savoir a Madame\index[indiv]{Blois (Marie de)} jusques a ce que il fust devers elle. Moi venu a Angiers au vespre, trouvé que verité estoit, et ne alé point devers Madame\index[indiv]{Blois (Marie de)} pour l'eure qu'il estoit trop tart.
                    
                Jeudi vegile saint Symon et saint Jude, je porté le seel de feu monseigneur en la chambre des comptes en la maison des Predicateurs, et le dit seel je enclos en un sac de toile et le lié tres bien et y fis mettre les signés du sire de Chasteaufromont\index[indiv]{Avoir (Pierre, sg. de Chateaufromont)}, de messire J. Pelerin\index[indiv]{Pelerin (Jean)}, du doyen d'Angiers\index[lieux]{Angiers}, de maistre J. le Begut\index[indiv]{Begut (Jean le)}, et de Thiebault Levraut\index[indiv]{Levraut (Thibaut)}. Et ledit seel ainsi enfermé je emporté: ce fu fait a matin.
                    
                Aprés disner je alé veoir Madame\index[indiv]{Blois (Marie de)}, et li fis la reverence, et dissimulé comme les aultres sanz li reveler la mort de monsegneur pour doubte du duc de Berry\index[indiv]{Berry (Jean de)}.
                    
                Samedi ensuivant Madame\index[indiv]{Blois (Marie de)} tint requestes, ignorant la mort de monsegneur et y fu messire Guillaume de Craon\index[indiv]{Craon (Guillaume de, sg. de la Ferté)}; et fu deliberé que les gens des trois pays qui estoient mandés au VI jour de novembre seroient contremandés par le conseil; et Madame\index[indiv]{Blois (Marie de)} en fu sachant et consentant.
                    
                La vegile de Toussains vint un messagé nommé Gastelet\index[indiv]{Gastelet (Messager)} apportant lettres de par monsegneur de Berri\index[indiv]{Berry (Jean de)} adressans aus genz du conseil de Madame\index[indiv]{Blois (Marie de)} et me apporta les dictes lettres. Je les ouvri present le sire de la Ferté\index[indiv]{Craon (Guillaume de, sg. de la Ferté)} et messire Regnaut Bresille\index[indiv]{Bresille (Regnaut)}.
                    
                Et puis m'en alé en la chambre des comptes; et y fu le sire de Chasteaufromont\index[indiv]{Avoir (Pierre, sg. de Chateaufromont)} et les autres du conseil et furent leues les lettres et en ladicte chambre laissiées. Comme devant fu deliberé de attendre le duc de Berry\index[indiv]{Berry (Jean de)}.
                    
                
            \chapter*{2. Marie de Blois apprend la mort de Louis Ier}
            
            \addcontentsline{toc}{chapter}{2. Marie de Blois apprend la mort de Louis Ier}
            
            Le jour des Mors aprés disner, Madame\index[indiv]{Blois (Marie de)} la royne sceut la mort de monsegneur le roy Loys\index[indiv]{Anjou (Louis Ier)}; moy et messire Guillaume de Craon\index[indiv]{Craon (Guillaume de, sg. de la Ferté)} et maistre J. le Begu\index[indiv]{Begut (Jean le)} et l'abbé de Saint Aubin, l'evesque d'Angiers\index[lieux]{Angiers}, le chantre et Thibauld Levraut\index[indiv]{Levraut (Thibaut)} la confortasmes ce que nous peusmes. Le sire de Chasteaufromont\index[indiv]{Avoir (Pierre, sg. de Chateaufromont)} vint veoir Madame\index[indiv]{Blois (Marie de)} et ploura comme une commere tres nicement sanz dire mot de reconfort.
                    
                Aprés fu deliberé que monsegneur de la Fierté\index[indiv]{Craon (Guillaume de, sg. de la Ferté)} escriroit a monsegneur de Berri\index[indiv]{Berry (Jean de)} comment Madame savoit ces nouvelles et li recommendoit soi et son estat.
                    
                Vendredi quart jour de novembre vint l'arcediacre de Chasteau de Ler de Paris\index[lieux]{Paris}; rapporta que conseil avoit esté tenu a Paris pour Madame\index[indiv]{Blois (Marie de)} ou avoient esté messire P. d'Orgemont\index[indiv]{Orgemont (Pierre d')}, messire Ernault de Corbic\index[indiv]{Corbic (Ernault de)}, maistre J. Canard\index[indiv]{Carnard (Jean)}, maistre Oudart de Molins\index[indiv]{Molins (Oudart de)}; conclus fu par eulx que a Madame\index[indiv]{Blois (Marie de)} de droit et de coustume apartient la garde de messegneurs ses enfans et le bail de leur terres se elle le veult prendre.
                    
                
            \chapter*{3. Jean Le Fèvre nommé chancelier de Marie de Blois}
            
            \addcontentsline{toc}{chapter}{3. Jean Le Fèvre nommé chancelier de Marie de Blois}
            
            Samedi ensuivant, V jour de novembre, Madame\index[indiv]{Blois (Marie de)} me retint son chancelier, present messire Guillaume de Craon segneur de la Ferté Bernard\index[indiv]{Craon (Guillaume de, sg. de la Ferté)}, et maistre Jehan Haucepié\index[indiv]{Haucepié (Jean)} a qui lettre en fu commandée. Et fis serment in verbo pontificis et sacerdotis, de la servir loyalment en office de chancelier, et la conseiller contre toulx exceptés le pape et le roy, et son honneur et pourfit garder et son blasme et damage eschever; mais elle ne me bailla point son seel encore.
                    
                
            \chapter*{4. La lettre du duc de Berry}
            
            \addcontentsline{toc}{chapter}{4. La lettre du duc de Berry}
            
            Lundi VIIe jour de novembre monsegneur de Berri\index[indiv]{Berry (Jean de)} escript au conseil que monsegneur de Bourgongne\index[indiv]{Bourgogne (Philippe le Hardi)} se estoit excusé de venir devers Madame\index[indiv]{Blois (Marie de)}; mes li monsegneur de Berri\index[indiv]{Berry (Jean de)} y seroit le samedi ensuivant.
                    
                
        
        \printindex[indiv]
        \addcontentsline{toc}{chapter}{Index des noms de personnes}
        \printindex[lieux]
        \addcontentsline{toc}{chapter}{Index des noms de lieux}
        
        %% Table des matières
        
        \tableofcontents
        
        \end{document}