\documentclass{beamer}
\usepackage[french]{babel}
\usepackage{inputenc}
\usetheme{metropolis}
\usecolortheme{cormorant}
\usepackage{minted}
\usepackage {ifthen}
\newboolean{printCode}
\setboolean{printCode}{true}
% \usemintedstyle{trac}
\usepackage{xcolor}
\usepackage{hyperref}
\usepackage{graphicx}
\usepackage{ulem}  % texte barré

% texte surligné
\usepackage{soul}
\usepackage{color}
\newcommand{\hilight}[1]{\colorbox{lightgray}{#1}}

\title{Technique et chaîne de publication électronique avec XSLT (7/7)}
\date{2023, 9 janv. - 20 fev.}
\author{Jean-Damien Généro}
\institute{École nationale des chartes -- M2 TNAH}

\begin{document}
    \maketitle

    \section{XSLT. Transformation vers \LaTeX}

    \begin{frame}{XSLT/ Méthode texte (1/2)}
        \Large
        \begin{itemize}
            \item Même en-tête XSL que pour la transformation vers HTML.
            \bigskip
            \item Transformation directe de XML à \LaTeX{} non-prise en charge par XSLT \textbf{$\rightarrow$ utiliser la méthode \texttt{text} dans le \texttt{<xsl:output/>}}.
            \bigskip
            \item Conséquence $\rightarrow$ les commandes \LaTeX{} ne sont pas reconnues comme telles et traitées comme du texte.
        \end{itemize}
    \end{frame}

    \begin{frame}[fragile]{XSLT/ Méthode texte (2/2). Template \LaTeX{} basique}
        \begin{minted}{tex}
\documentclass[]{book}
\usepackage[T1]{fontenc}
\usepackage[french]{babel}
\title{}
\author{}
\date{}

\begin{document}
\maketitle
% votre texte
\tableofcontents
\addcontentsline{toc}{chapter}{Table des matières}
\end{document}
        \end{minted}
    \end{frame}
\end{document}
