\documentclass{beamer}
\usepackage[french]{babel}
\usepackage{inputenc}
\usetheme{metropolis}
\usecolortheme{cormorant}
\usepackage{minted}
\usepackage {ifthen}
\newboolean{printCode}
\setboolean{printCode}{true}
% \usemintedstyle{trac}
\usepackage{xcolor}
\usepackage{hyperref}
\usepackage{graphicx}
\usepackage{ulem}  % texte barré

% texte surligné
\usepackage{soul}
\usepackage{color}
\newcommand{\hilight}[1]{\colorbox{lightgray}{#1}}

\title{Technique et chaîne de publication électronique avec XSLT (3/7)}
\date{2023, 9 janv. - 13 fev.}
\author{Jean-Damien Généro}
\institute{École nationale des chartes -- M2 TNAH}

\begin{document}
    \maketitle

    \section{XSLT. Les attributs de \texttt{<xsl:apply-templates/>}}

    \begin{frame}[fragile]{XSLT/ Rappel : l'instruction \texttt{<xsl:apply-templates/>}}
        \Large
        \begin{itemize}
            \item Permet d'appliquer les règles aux n\oe uds enfants du n\oe ud courant.
            \item Sans le \texttt{<xsl:apply-templates/>}, XSLT ne chercherait pas de règle pour les enfants de <TEI> dans l'exemple ci-dessous :
        \end{itemize}
        % noeud courant = celui dans @select de <xsl:template> (fixe)
        % noeud de contexte = celui sur le quel opère la règle (variable)
        \begin{minted}{xml}
<xsl:template match="/">
    <TEI>
        <xsl:apply-templates/>
    </TEI>
</xsl:template>
        \end{minted}
    \end{frame}

    \begin{frame}[fragile]{XSLT/ \texttt{<xsl:apply-templates/>}, attribut \texttt{@select} (1/3)}
        \Large
        \begin{itemize}
            \item \texttt{@select} permet d'appliquer les règles uniquement aux n\oe uds enfants sélectionnés.
            \bigskip
            \item Ex. $\rightarrow$ inverser \texttt{<teiHeader/>} et \texttt{<text/>} (\textit{cf.} fichier) : 
        \end{itemize}
        \normalsize
        \begin{minted}{xml}
<xsl:template match="/">
    <TEI>
        <xsl:apply-templates select="//text"/>
        <xsl:apply-templates select="//teiHeader"/>
    </TEI>
</xsl:template>
        \end{minted}
    \end{frame}

    \begin{frame}{XSLT/ \texttt{<xsl:apply-templates/>}, attribut \texttt{@mode} (2/3)}
        \Large
        \begin{itemize}
            \item \texttt{@mode} permet d'appliquer des règles différentes \textbf{à un même élément XML} en fonction de son emplacement ou de son contenu (= son \og mode \fg).
            \bigskip
            \item Le \texttt{@mode} doit être présent et dans le \texttt{<xsl:apply-templates/>} et dans le \texttt{<xsl:template/>} avec la même valeur (= le nom du mode).
        \end{itemize}
    \end{frame}

    \begin{frame}[fragile]{XSLT/ \texttt{<xsl:apply-templates/>}, attribut \texttt{@mode} (3/3)}
        \Large
        \begin{itemize}
            \item Exemple d'utilisation de \texttt{@mode} : créer une table des matières (\textit{cf.} fichier).
        \end{itemize}
        \normalsize
        \begin{minted}{xml}
<xsl:template match="//body">
    <xsl:copy>
        <xsl:apply-templates />
        <div>
            <head>Table des matières</head>
            <xsl:apply-templates mode="toc"/>
        </div>
    </xsl:copy>
</xsl:template>
        \end{minted}
    \end{frame}

\end{document}
