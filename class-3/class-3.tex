\documentclass{beamer}
\usepackage[french]{babel}
\usepackage{inputenc}
\usetheme{metropolis}
\usecolortheme{cormorant}
\usepackage{minted}
\usepackage {ifthen}
\newboolean{printCode}
\setboolean{printCode}{true}
% \usemintedstyle{trac}
\usepackage{xcolor}
\usepackage{hyperref}
\usepackage{graphicx}
\usepackage{ulem}  % texte barré

% texte surligné
\usepackage{soul}
\usepackage{color}
\newcommand{\hilight}[1]{\colorbox{lightgray}{#1}}

\title{Technique et chaîne de publication électronique avec XSLT (3/7)}
\date{2023, 9 janv. - 13 fev.}
\author{Jean-Damien Généro}
\institute{École nationale des chartes -- M2 TNAH}

\begin{document}
    \maketitle
    
    \section{XSLT. Variables et paramètres}

    \begin{frame}{XSLT/ Différence entre \texttt{variable} et \texttt{param}}
        \Large
        \begin{itemize}
            \item Rappel $\rightarrow$ variable : nom (unique) + valeur (statique ou dynamique).
            \bigskip
            \item Dans XSLT, portée globale (premier niveau) ou locale (intégrée à un \texttt{<xsl:template/>}) ;
            \bigskip
            \item \texttt{<xsl:variable/>} : \textbf{valeur fixe} ;
            \bigskip
            \item \texttt{<xsl:param/>} : \textbf{valeur fixe ou dynamique}, peut être pris en argument par le processeur (ex : saxon dans lxml).
        \end{itemize}
    \end{frame}

    \begin{frame}[fragile]{XSLT/ \texttt{<xsl:variable/>} (1/2)}
        \Large
        \begin{itemize}
            \item Attributs : \texttt{@name} \textit{(req.)} et \texttt{@select} \textit{(opt.)} ;
            \bigskip
            \item Deux manières de définir la valeur de \texttt{<xsl:variable/>} :
            \begin{itemize}
            \Large
                \item Par le contenu de l'\texttt{@select} (expression XPath) ;
                \item Par le contenu de \texttt{<xsl:variable/>} (texte ou règle).
            \end{itemize}
        \end{itemize}
        \normalsize
        \begin{minted}{xml}
<xsl:variable name="x" select="Xpath"/>
<xsl:variable name="y">[règles]</xsl:variable>
        \end{minted}
    \end{frame}

    \begin{frame}[fragile]{XSLT/ \texttt{<xsl:variable/>} (2/2)}
        \Large
        \begin{itemize}
            \item  Pour appeler une variable : \texttt{\$} + \texttt{@name} dans le \texttt{@select} d'un \texttt{<value-of/>}
        \end{itemize}
        \normalsize
        \begin{minted}{xml}
<xsl:template math="/">
    <xsl:value-of select="$x"/>
    <xsl:value-of select="$y"/>
</xsl:template>
        \end{minted}
    \end{frame}

    \begin{frame}{XSLT/ \texttt{<xsl:param/>}}
        \Large
        \begin{itemize}
            \item Fonctionne de la même manière que \texttt{<xsl:variable/>}.
        \end{itemize}
    \end{frame}

    \section{XPath/ Les fonctions}

    \begin{frame}{XPath/ Fonctions}
        \Large
        \begin{itemize}
            \item Les fonctions peuvent être utilisées dans des expressions XPath ou dans des prédicats :
            \begin{itemize}
            \Large
                \item Expression $\rightarrow$ \texttt{count(./p)}
                \item Prédicat (\texttt{[ ]}) $\rightarrow$ \texttt{./[count(p)]}
            \end{itemize}
        \end{itemize}
    \end{frame}
\end{document}
