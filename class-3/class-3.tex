\documentclass{beamer}
\usepackage[french]{babel}
\usepackage{inputenc}
\usetheme{metropolis}
\usecolortheme{cormorant}
\usepackage{minted}
\usepackage {ifthen}
\newboolean{printCode}
\setboolean{printCode}{true}
% \usemintedstyle{trac}
\usepackage{xcolor}
\usepackage{hyperref}
\usepackage{graphicx}
\usepackage{ulem}  % texte barré

% texte surligné
\usepackage{soul}
\usepackage{color}
\newcommand{\hilight}[1]{\colorbox{lightgray}{#1}}

\title{Technique et chaîne de publication électronique avec XSLT (3/7)}
\date{2023, 9 janv. - 13 fev.}
\author{Jean-Damien Généro}
\institute{École nationale des chartes -- M2 TNAH}

\begin{document}
    \maketitle

    \section{XSLT. Les attributs de \texttt{<xsl:apply-template/>}}

    \begin{frame}[fragile]{XSLT/ Rappel : l'instruction \texttt{<xsl:apply-template/>}}
        \Large
        \begin{itemize}
            \item Permet d'appeler les règles aux n\oe uds enfants du n\oe ud courant.
            \item Sans le \texttt{<xsl:apply-template/>}, XSLT ne chercherait pas de règle pour les enfants de <TEI> dans l'exemple ci-dessous :
        \end{itemize}
        % noeud courant = celui dans @select de <xsl:template> (fixe)
        % noeud de contexte = celui sur le quel opère la règle (variable)
        \begin{minted}{xml}
<xsl:template match="/">
    <TEI>
        <xsl:apply-templates/>
    </TEI>
</xsl:template>
        \end{minted}
    \end{frame}

    \begin{frame}[fragile]{XSLT/ \texttt{<xsl:apply-template/>}, attribut \texttt{@select} (1/2)}
        \Large
        \begin{itemize}
            \item \texttt{@select} permet d'appliquer les règles uniquement aux n\oe uds enfants sélectionnés.
            \bigskip
            \item Exemple $\rightarrow$ inverser le \texttt{<teiHeader/>} et le \texttt{<text/>} : 
        \end{itemize}
        \normalsize
        \begin{minted}{xml}
<xsl:template match="/">
    <TEI>
        <xsl:apply-templates select="//text"/>
        <xsl:apply-templates select="//teiHeader"/>
    </TEI>
</xsl:template>
        \end{minted}

        % exemple dans Oxygen :
        % <?xml version="1.0" encoding="UTF-8"?>
        % <xsl:stylesheet xmlns:xsl="http://www.w3.org/1999/XSL/Transform"
        %     xmlns:xs="http://www.w3.org/2001/XMLSchema"
        %     xmlns:tei="http://www.tei-c.org/ns/1.0"
        %     xpath-default-namespace="http://www.tei-c.org/ns/1.0"
        %     xmlns="http://www.tei-c.org/ns/1.0"
        %     exclude-result-prefixes="xs tei"
        %     version="2.0">
        %     <xsl:output method="xml" indent="yes" omit-xml-declaration="no"/>
        %     <xsl:template match="/">
        %         <TEI>
        %             <xsl:apply-templates select="//text"/>
        %             <xsl:apply-templates select="//teiHeader"/>
        %         </TEI>
        %     </xsl:template>
        %     <xsl:template match="//teiHeader">
        %         <xsl:copy-of select="."/>
        %     </xsl:template>
        %     <xsl:template match="//text">
        %         <xsl:copy-of select="."/>
        %     </xsl:template>
        % </xsl:stylesheet>
    \end{frame}

    \begin{frame}{XSLT/ \texttt{<xsl:apply-template/>}, attribut \texttt{@mode} (2/2)}
        \Large
        \begin{itemize}
            \item \texttt{@mode} permet d'appliquer des règles différentes à un même élément XML en fonction de son emplacement ou de son contenu (= son \og mode \fg).
        \end{itemize}
    \end{frame} 

\end{document}
