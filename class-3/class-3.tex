\documentclass[a4paper]{article}
\usepackage[french]{babel}
\usepackage[T1]{fontenc}

\title{\textsc{École nationale des chartes}

\bigskip

Technique et chaîne de publication électronique avec XSLT}
\author{Jean-Damien Généro}
\date{2023, 9 janv. - 13 fev.}

\begin{document}
\maketitle

\section*{Informations générales}

\begin{itemize}
    \item Cours en ligne : https://github.com/jeandamien-genero/xslt-class
    \item Contact : jean-damien.genero [at] cnrs [.] fr
\end{itemize}

\bigskip

\section*{Programme}

\renewcommand{\arraystretch}{1.5}
\begin{center}
    \begin{tabular}[h]{|p{1.5cm} p{9.6cm}}
        \textbf{9 janv.} & \textbf{1.} Introduction. \texttt{XPath}. Premières notions de \texttt{XSLT}.\\
        \textbf{16 janv.} & \textbf{2.} \texttt{XSLT} : règles basiques.  \\
        \textbf{23 janv.} & \textbf{3.} \texttt{XSLT} : \texttt{<xsl:apply-template/>}, variables et fonctions \texttt{XPath}. \\
        \textbf{30 janv.} & \textbf{4.} \texttt{XSLT} : règles complexes (conditions et boucles). \\
        \textbf{6 fev.} & \textbf{!}  Contrôle continu.
        
        \textbf{5.} Transformation de \texttt{XML}  vers \texttt{XSLT}. \\
        \textbf{13 fev.} & \textbf{6.} Transformation de \texttt{XML}  vers \texttt{HTML}. \\
        \textbf{20 fev.} & \textbf{7.} Transformation de \texttt{XML}  vers \texttt{LaTeX}. \\
    \end{tabular}
\end{center}

\bigskip

\section*{Objectifs}
      \begin{itemize}
          \item \texttt{XPath}
          \begin{itemize}
              \item Naviguer dans un arbre \texttt{XML}
              \item Manipuler les principales fonctions \texttt{XPath}
          \end{itemize}
          \item \texttt{XSLT}
          \begin{itemize}
              \item Manipuler les règles (\textit{templates}) basiques
              \item Manipuler les variables, conditions et boucles
          \end{itemize}
          \item Édition numérique
          \begin{itemize}
              \item Transformer un document \texttt{XML} en un autre document \texttt{XML}
              \item Transformer un document \texttt{XML} en un document \texttt{HTML}
              \item Transformer un document \texttt{XML} en un document \texttt{LaTeX}
          \end{itemize}
      \end{itemize}

\end{document}
