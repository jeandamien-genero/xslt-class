\documentclass{beamer}
\usepackage[french]{babel}
\usepackage{inputenc}
\usetheme{metropolis}
\usecolortheme{cormorant}
\usepackage{minted}
\usepackage {ifthen}
\newboolean{printCode}
\setboolean{printCode}{true}
% \usemintedstyle{trac}
\usepackage{xcolor}
\usepackage{hyperref}
\usepackage{graphicx}
\usepackage{ulem}  % texte barré

% texte surligné
\usepackage{soul}
\usepackage{color}
\newcommand{\hilight}[1]{\colorbox{lightgray}{#1}}

\title{Technique et chaîne de publication électronique avec XSLT (6/7)}
\date{2024, 3 dec. - 2025, 22 janv.}
\author{Jean-Damien Généro}
\institute{École nationale des chartes -- M2 TNAH}

\begin{document}
    \maketitle

    \section{XSLT. Transformation vers HTML}

    \begin{frame}{XSLT/ Méthode HTML (1/2)}
        \Large
        \begin{itemize}
            \item Dans l'en-tête XSL : les \texttt{@xmlns:tei}, \texttt{@xmlns} et \texttt{@exclude-result-prefixes} ne sont plus nécessaires ;
            \bigskip
            \item Dans \texttt{<xsl:output/>} : ajouter \texttt{@method='html'}, \texttt{@omit-xml-declaration} n'est plus nécessaire.
        \end{itemize}
    \end{frame}

    \begin{frame}[fragile]{XSLT/ Méthode HTML (2/2). Template HTML basique}
        \begin{minted}{html}
<html lang="fr">
    <head>
        <meta http-equiv="Content-Type" 
           content="text/html; charset=UTF-8">
        <!-- le 1er <meta> est ajouté par XSLT -->
        <title><!-- doc title --></title>
        <meta name="description" content="#"/>
        <meta name="author" content="#"/>
        <!-- <link rel="stylesheet" href="#"/> -->
    </head>
    <body>
        <!-- some text here -->
        <!-- <script src="#"></script> -->
    </body>
</html>
        \end{minted}
    \end{frame}
    
    \begin{frame}{XSLT/ Instruction \texttt{call-template} (1/2)}
        \Large
        \begin{itemize}
            \item \textbf{\texttt{<xsl:call-template name="\#"/>}} ;
            \item Permet d'appeler autant de fois qu'on le veut une template dans un \texttt{<xsl:template/>}
            \item Cas pratiques :
            \begin{itemize}
            \Large
                \item Un \texttt{call-template} par \texttt{result-document} ;
                \item Peut permettre de stocker des parties du document de sortie (header, footer) au même titre que les variables.
            \end{itemize}
        \end{itemize}
    \end{frame}

    \begin{frame}[fragile]{XSLT/ Instruction \texttt{call-template} (2/2). Exemple.}
        \begin{minted}{html}
<xsl:template match="/">
    <xsl:call-template name="name1"/>
</xsl:template>

<xsl:template name="name1">
    <xsl:result-document href="#">
        <html>
            ...
        </html>
    </xsl:result-document>
</xsl:template>
        \end{minted}
    \end{frame}
\end{document}
